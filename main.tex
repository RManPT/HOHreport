% M.Sc. Dissertation Template
%	A work by Ana Lopes, based on a template provided by the Department
%   of Electrical and Computer engineering form The university of Coimbra
%   and using the rules provided by the Polytechnic Institute of Tomar.
%
%	This template should compile into a formally acceptable Master's
%	Dissertation. It should even compile without you having to actually
% 	having to do anything (aside from, you know, calling pdflatex), so
%	you can get an idea of how it will look.
%
%	I know reading code sucks, but I strongly advise you to take a close
%	look at this file. Really, it's not that long. It's 150 lines and
%	most of it is cosmetic whitespace.


% Preamble
\documentclass[a4paper, twoside, 12pt]{report}

% % -------------------- New packages  ----------------------------
% \usepackage{multirow}
%\usepackage{changepage}
%\usepackage{titlesec}
\usepackage{amsmath}
\usepackage{float}
\usepackage{fancyhdr}

\usepackage[linesnumbered,ruled,vlined]{algorithm2e}
\usepackage[noend]{algpseudocode}
\algnewcommand\algorithmicforeach{\textbf{for each}}
\algdef{S}[FOR]{ForEach}[1]{\algorithmicforeach\ #1\ \algorithmicdo}

\usepackage{array}
\newcolumntype{P}[1]{>{\centering\arraybackslash}p{#1}}

\setcounter{secnumdepth}{3}

\usepackage{textcomp}
\usepackage{gensymb}


% % ------------------------------------------------

% Includes
		% UTF-8 encoding, so that yogoou can use characters like ç and ã

\usepackage[T1]{fontenc}				% Same, but for output encoding
\usepackage[utf8]{inputenc}		
\usepackage[portuges]{babel}	% Still related to the above
\usepackage{acronym}					% List of acronyms
\usepackage{textcomp} 					% Extra characters
\usepackage{graphicx} 					% \includegraphics{}, the most common command to include images in figures
\usepackage{titlesec}					% To manually format the chapter titles
\usepackage[left=3cm,right=2.5cm,top=2.5cm,bottom=2.5cm]{geometry} % Margins, as dictated by the rules
%\usepackage[nottoc,numbib]{tocbibind} 	% Hyperlinks in table of contents, useful for navigation
\usepackage[section]{placeins}			% \FloatBarrier, a useful command when your figures are trying to run away
\usepackage{caption}					% For captioning figures
\usepackage{subcaption}					% Subfigures (the subfigure package is deprecated and should not be used)
\usepackage[toc,page]{appendix}			% Appendices
\usepackage{pdfpages}					% Useful when your appendix is a pre-compiled PDF, such as a whole paper
\usepackage{url}						% Useful when one wants to include URLs in the text
\usepackage[
      colorlinks=true,    			%no frame around URL
      urlcolor=black,    			%no colors
      menucolor=black,    			%no colors
      linkcolor=black,    			%no colors
      citecolor=black,    			%no colors
      bookmarks=true,    			%tree-like TOC
      bookmarksopen=true,    		%expanded when starting
      bookmarksnumbered=true, 		%Put section numbers in bookmarks
      hyperfootnotes=true,    		%no referencing of footnotes, does not compile
      pdfpagemode=UseOutlines,    	%show the bookmarks when starting the pdf viewer
      plainpages=false, 			%solve problem ``destination with the same identifier'' warning
      pdfpagelabels				 	%solve problem ``destination with the same identifier'' warning
]{hyperref} 							% So that our citations look good and still work as links
\usepackage{epigraph}					% For your inspirational quote
\usepackage{etoolbox}
\usepackage{enumitem}
\usepackage{listings,xcolor}
\usepackage{algpseudocode, algorithm2e}
\usepackage{algcompatible}
\usepackage{booktabs}
\usepackage{multirow}
\usepackage{emptypage}
%\usepackage{subfigure}
%\usepackage{fontspec}

\setlength{\parindent}{2em}

\setlength{\headheight}{16pt}
\renewcommand{\baselinestretch}{1.3}	% 1.5 line spacing, as mandated by the rules
%\titleformat{\chapter}[hang] 			% Smaller chapter titles
%{\normalfont\huge\bfseries}{\thechapter}{1em}{}

% Your info goes here
\newcommand{\thesistitle}{Template Dissertação do MEI-IdC}			% Your work's title
\newcommand{\myname}{Ana Cristina Barata Pires Lopes}				% Your name
\newcommand{\statedate}{Tomar, dezembro 2017}					% The date, usually "Place, Month Year"
\newcommand{\supervisorname}{Professor Doutor Einstein}		% Your supervisor's name
\newcommand{\cosupervisorname}{Professora Doutora Marie Curie}	% Your co-supervisor's name, if any.


\renewcommand\lstlistingname{}
\renewcommand\lstlistlistingname{Algorithms}

%\DeclareUnicodeCharacter{00A0}{~}

\makeatletter
\renewcommand*{\cleardoublepage}{\clearpage\if@twoside \ifodd\c@page\else
\hbox{}%
\thispagestyle{empty}%
\newpage%
\if@twocolumn\hbox{}\newpage\fi\fi\fi}
\makeatother

\pagestyle{plain}
\addto\captionsportuges{% Replace "english" with the language you use
  \renewcommand{\contentsname}%
    {ÍNDICE}%
}
\addto\captionsportuges{% Replace "english" with the language you use
  \renewcommand{\listtablename}%
    {ÍNDICE DE TABELAS}%
}
\addto\captionsportuges{% Replace "english" with the language you use
  \renewcommand{\listfigurename}{ÍNDICE DE FIGURAS}
}

\renewcommand\appendixtocname{Apêndice}
\renewcommand\appendixpagename{Apêndice}


% MAIN DOCUMENT
\begin{document}
\pagestyle{headings}
\pagenumbering{roman}

% TITLE PAGES
% Uncomment this line when you have your cover ready. An MSWord template is available at that folder.
% You should edit it in MSWord, and then export it into PDF, so we can neatly import it here.
\includepdf[pages={-}]{images/cover/subcapa.pdf}
% Blank page
\newpage
\thispagestyle{empty}
\mbox{}
% Title page 1
%\input{title_page1}
% Blank page
%\newpage
%\thispagestyle{empty}
%\mbox{}

% Acknowledgements
\titleformat{\chapter}[display]{\rmfamily\Large\bfseries}{\thechapter}{0.5ex}{\centering}[\vspace{-0.5ex}\rule{\textwidth}{0.3pt}]
\chapter*{AGRADECIMENTOS}
\addcontentsline{toc}{chapter}{Agradecimentos}
\input{acknowledgements}
% You can add blank pages here, if you like
\newpage\null\thispagestyle{empty}\newpage%Comentar esta linha se não for preciso página em branco


% RESUMO
\titleformat{\chapter}[display]{\rmfamily\Large\bfseries}{\thechapter}{0.5ex}{\centering}[\vspace{-0.5ex}\rule{\textwidth}{0.3pt}]
\chapter*{RESUMO}
\addcontentsline{toc}{chapter}{Resumo}
\input{resumo}
\newpage\null\thispagestyle{empty}\newpage %Comentar esta linha se não for preciso página em branco

%ABSTRACT
\titleformat{\chapter}[display]{\rmfamily\Large\bfseries}{\thechapter}{0.5ex}{\centering}[\vspace{-0.5ex}\rule{\textwidth}{0.3pt}]
\chapter*{ABSTRACT}
\addcontentsline{toc}{chapter}{Abstract}
\input{abstract}
% And here as well
\newpage\null\thispagestyle{empty}\newpage%Comentar esta linha se não for preciso página em branco

% INSPIRATIONAL QUOTE
% Setup
\setlength\epigraphwidth{12cm}
\setlength\epigraphrule{0pt}
\makeatletter
\patchcmd{\epigraph}{\@epitext{#1}}{\itshape\@epitext{#1}}{}{}
\makeatother
% Actual Quote
\vspace*{\fill}
\epigraph{"It's not when you get there, it's always the climb."}{}
{ ---  \textup{Robert A. Heinlein}}
\vspace*{\fill}
\newpage\null\thispagestyle{empty}\newpage

% TABLE OF CONTENTS

\titleformat{\chapter}[display]{\rmfamily\Large\bfseries}{\thechapter}{0.5ex}{\centering}[\vspace{-0.5ex}\rule{\textwidth}{0.3pt}]
\tableofcontents
\clearpage
% LIST OF ACRONYMS
\chapter*{ACRÓNIMOS}
\addcontentsline{toc}{chapter}{Acrónimos}
\input{list_acronyms}


\addcontentsline{toc}{chapter}{Índice de Figuras}
\listoffigures
\clearpage %\cleardoublepage %for openright
\addcontentsline{toc}{chapter}{Índice de Tabelas}
\listoftables
\clearpage %\cleardoublepage %for openright
% BODY
\newpage
\thispagestyle{empty}
\mbox{}

\fancyhead[LE,RE]{\slshape\rightmark}
\fancyhead[LO,RO]{\slshape\leftmark}
\fancyhead[RE,LO]{}
\pagestyle{fancy}
\titleformat{\chapter}[display]
    {\normalfont\huge\bfseries}{\chaptertitlename\ \thechapter}{20pt}{\Huge}
\titlespacing*{\chapter}{0pt}{0pt}{20pt}

\chapter{Introdução}
\pagenumbering{arabic}
\input{introduction}
	% Arabic numbering starts

% For each chapter, you should have a bit of code that looks like this:
% \label allows you to later \ref that chapter.
% \input includes a different .tex file, so that you can have you dissertation
% neatly partitioned into several files. I recommend one file per chapter.
\chapter{Estado da Arte}
\label{chap:state_of_the_art}
\input{state_of_the_art}

\chapter{Fundamentos}
\label{chap:background_material}
\input{background_material}

\chapter{Conclusões}
\input{conclusion_future_work}

% REFERENCES
% Edit the references.bib file to add your own references, that you can then
% \cite on your text.


\bibliographystyle{ieeetr}

\addcontentsline{toc}{chapter}{Bibliografia}
\bibliography{references}

%\clearpage %\cleardoublepage %for openright
\renewcommand{\thesection}{Apêndices \Roman{section}}

\begin{appendices}

\end{appendices}

\end{document}
